\documentclass[UTF8]{ctexart}
\title{视频用户体验评估——函数关系的确立}
\begin{document}
\maketitle
\section{摘要}
\section{问题重述}
\subsection{问题的提出}
随着科技的进步——无线宽带网络的升级、智能用户端的便携化、各类视频直播软件的普及,愈来愈多的用户选择在移动智能终端上使用应用客户端(APP)观看网络视频。网络视频是一种基于传输控制协议(TCP——一种面向连接的、可靠的、基于字节流的传输层通信协议,通常使用一个校验和函数来检验数据是否有错误)的视频传送及播放。

其中影响用户体验的两个关键指标——初始缓冲等待时间和视频播放过程中的卡顿缓冲时间,主要使用初始缓冲时延(sLoading)和卡顿占比(sStalling)(卡顿时长的占比=卡顿时长/视频播放时长)来定量地评价。研究数据表明,影响两个关键指标的主要因素有四——初始缓冲峰值速率、播放阶段平均下载速率、端到端环回时间(E2E RTT)以及视频参数。但目前以上四个因素与两个关键指标之间的关系未知。

请依据题目附件所提供的实验数据建立用户体验变量(初始缓冲时延,卡顿时长占比)与网络侧变量(初始缓冲峰值速率,播放阶段平均下载速率,E2E RTT)之间的函数关系。
\subsection{问题的分析}
由题意可知,本题的目的是为了建立一个模型来判定影响视频用户体验的两个关键指标与四个相关影响因素的函数关系。由于本题有多个自变量和因变量,所以需要得出一个多元向量值函数。已知函数的输入值和输出值,需要运用神经网络的相应知识倒推出所求函数。
\section{模型假设及符号说明}
\subsection{模型的假设}
\subsection{符号说明}
\section{模型的建立}
\section{模型的求解及结果分析}
\subsection{模型的求解方法}
\subsection{结果分析}
\section{模型的检验}
\section{模型的优缺点及改进方向}
\subsection{模型的优点}
\subsection{模型的缺点}
\subsection{模型的改进方向}
\section{参考文献}
\section{附录}
\end{document}