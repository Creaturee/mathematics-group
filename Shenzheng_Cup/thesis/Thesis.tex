\documentclass[UTF8]{ctexart}
\usepackage{amsmath}
\title{视频用户体验评估——函数关系的确立}
\begin{document}
\maketitle
\section{摘要}
\section{问题重述}
\subsection{问题的提出}
\paragraph{}
随着科技的进步——无线宽带网络的升级、智能用户端的便携化、各类视频直播软件的普及,
愈来愈多的用户选择在移动智能终端上使用应用客户端(APP)观看网络视频,
这就是我们常说的流媒体业务。
\paragraph{}
流媒体的典型传输方式为:连续采样的多媒体信息经编码压缩后,从源端连续、实时地发送到网络, 
接收端从网络收到部分数据后就可以开始解码播放,而后续接 收的数据则持续不断地存入本地缓存,
形成可连续播放的媒体流。\cite{1}
可以看出,相对传统E-mail等传统网络业务,实时流媒体对于实时性、 网络带宽、容错性、媒体同步、播放平滑性等方面均有严格要求。
\paragraph{}
相比UDP和IP等传输协议,TCP具有拥塞控制,丢包重传等机制,因而更加可靠。
因此,目前的流媒体传输技术主要基于TCP传输控制协议。
\paragraph{}
流媒体业务模型中,影响用户体验的两个关键指标——初始缓冲等待时间和视频播放过程中的卡顿缓冲时间,
主要使用初始缓冲时延(sLoading)和卡顿占比(sStalling)(卡顿时长的占比=卡顿时长/视频播放时长)来定量地评价。
影响两个关键指标的主要因素包括初始缓冲峰值速率、播放阶段平均下载速率、端到端环回时间(E2E RTT)以及视频参数。
但目前以上这些因素与两个关键指标之间的关系未知。
\paragraph{}
显然,建立用户体验变量(初始缓冲时延,卡顿时长占比)
与网络侧变量(初始缓冲峰值速率,播放阶段平均下载速率,E2E RTT)之间的函数关系对于改善用户体验,
提高流媒体业务服务水平具有重要意义。
\subsection{问题的分析}
\paragraph{}
由题意可知,本题的目的是为了建立一个多元函数模型来确定影响视频用户体验的两个关键指标与视频相关参数的关系。
由于本题目数据较多,便于计算机进行统计和分析,故我们采用两种不同的策略进行建模。
\paragraph{}
第一种是白箱(class box)方法,即传统意义上的数学建模方法——通过查找文献获取问题相关知识,
分析其机理并建立数学模型,最终通过数据确定参数,并分析模型准确度和误差。在此,最大的挑战便
在于查找文献确定端到端环回时间(RTT)在TCP丢包重传和拥塞控制机制中所起到的作用,并确定其对
客户端播放产生的影响。

\paragraph{}
另一种则是黑箱(black box)方法,将数据内部关系视为黑箱,仅仅从输入和输出数据的角度进行观测,
并通过统计学方法拟合出变量间函数关系。
值得一提的是,以神经网络为代表的机器学习技术近年来在工程和科研计算中得到了越来越广泛的应用,
其对任意非线性多元函数进行拟合的能力相较于传统方法更加简单且容易推广。
同时,在拥有大量数据的基础上,神经网络可以达到相当的准确度,这也是我们采取这一方法的主要原因。

\section{模型假设及符号说明}

\subsection{符号说明}
初始缓冲峰值速率:$v_{max} $
E2E RTT: $t_{R}$
播放阶段平均速率: $v^{_} $
初始缓冲时延: $t_{0} $
初始缓冲下载数据量: $s_{0} $
视频码率: $K $
播放阶段总时长: $t_{all} $
播放时长: $t_{p} $
卡顿时长: $t_{s} $

\subsection{模型的假设}
\paragraph{1.}
TCP传输协议带宽大于视频码率。这是TCP用于流媒体业务的基本要求。
\paragraph{2.}
TCP客户端和服务端的缓冲区大小能够容TCP的视频帧, 而不会在TCP接收缓冲区溢出。
考虑到目前普及的个人终端设备已经具有较高性能,因此该条件易满足。
\paragraph{3.}
视频码率固定不变,由题目数据可以看出该条件默认成立。
\paragraph{4.}
假设TCP协议将视频数据分包发送时,每个包(package)的大小相同。
\paragraph{5.}
假设客户端的视频连续播放,即在视频播放中用户没有进行手动快进或者暂停操作。
由题目数据可以看出该条件默认成立。
\paragraph{6.}




\section{模型的建立}

\subsection{分析方法}
若想对视频传输系统建立数学模型,我们首先要理解的就是就是基于TCP和移动网络的在线视频业务模型。
在线视频技术就是我们常说的流媒体技术。@刘效妍


视频数据量=视频码率*播放时长

初始缓冲时延=初始缓冲下载数据量/缓冲时段平均速率 --1

初始缓冲下载数据量+播放阶段总时长*播放阶段平均速率*>=视频数据量--2

不出现卡顿的条件:(视频数据量-初始缓冲下载数据量)/播放阶段平均速率 < 视频播放时间

卡顿时间 = (视频数据量-初始缓冲下载数据量)/播放阶段平均速率 - 视频播放时间

\subsection{统计方法}
神经网络

\section{模型的求解及结果分析}

视频缓冲阶段峰值下载速率和平均下载速率接近正比

\subsection{模型的求解方法}
\subsection{结果分析}

\section{模型的检验}
\section{模型的优缺点及改进方向}
\subsection{模型的优点}
\subsection{模型的缺点}
\subsection{模型的改进方向}
\section{参考文献}
\section{附录}
\end{document}