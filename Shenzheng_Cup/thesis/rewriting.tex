\documentclass[UTF8]{ctexart}
\usepackage{amsmath}
\title{视频用户体验评估——函数关系的确立}
\usepackage{amsmath}
\begin{document}
\maketitle
\section{摘要}
\section{问题重述}
\subsection{问题的提出}
随着科技的进步——无线宽带网络的升级、智能用户端的便携化、各类视频直播软件的普及,愈来愈多的用户选择在移动智能终端上使用应用客户端(APP)观看网络视频。网络视频是一种基于传输控制协议(TCP)的视频传送及播放。

其中影响用户体验的两个关键指标——初始缓冲等待时间和视频播放过程中的卡顿缓冲时间,主要使用初始缓冲时延(sLoading)和卡顿占比(sStalling)(卡顿时长的占比=卡顿时长/视频播放时长)来定量地评价。研究数据表明,影响两个关键指标的主要因素有四——初始缓冲峰值速率、播放阶段平均下载速率、端到端环回时间(E2E RTT)以及视频参数。但目前以上四个因素与两个关键指标之间的关系未知。

请依据题目附件所提供的实验数据建立用户体验变量(初始缓冲时延,卡顿时长占比)与网络侧变量(初始缓冲峰值速率,播放阶段平均下载速率,E2E RTT)之间的函数关系。
	
\section{模型假设及符号说明}
\subsection{模型的假设}
\subsection{符号说明}
1.初始缓冲峰值速率:$v_{max} $

2. E2E RTT: $t_{R}$

3.播放阶段平均速率: $ \bar{v} $

4.初始缓冲时延: $t_{0} $

5.初始缓冲下载数据量: $s_{0} $

6.视频码率: $K $

7.播放阶段总时长: $t_{all} $

8.播放时长: $t_{p} $

9.卡顿时长: $t_{s} $

\section{问题的分析}
由题意可知,本题的目的是为了建立一个多元函数模型来确定影响视频用户体验的两个关键指标与视频相关参数的关系。

由于本题目数据较多,为了便于计算机进行统计和分析,我们采用两种不同的策略进行建模:第一种是白箱(Class Box)方法,即传统意义上的数学建模方法——通过查找文献获取问题相关知识,分析其机理并建立数学模型,最终通过数据确定参数,并分析模型准确度和误差;另一种则是黑箱(Black Box)方法,将数据内部关系视为黑箱,仅仅从输入和输出数据的角度进行观测,并通过统计学方法拟合出变量间函数关系。

已知黑箱方法的基础为大量观测数据,而在该问题中,我们得到了数量可观的实际数据,这是促使我们采用这一方法的主要原因。同时,值得一提的是,以神经网络为代表的机器学习技术近年来在工程和科研计算中得到了越来越广泛的应用,其对任意非线性多元函数进行拟合的能力相较于传统方法更具优势,这是我们采取这一方法的另一原因。
\section{模型的建立}
\subsection{模型的准备}
若想对视频传输系统建立数学模型,我们首先要理解的就是就是基于TCP和移动网络的在线视频业务模型。
在线视频技术就是我们常说的流媒体技术。
\subsubsection{传输控制协议(TCP)}
传输控制协议(Transmission Control Pritocol)是一种面向连接的、相对IP而言可靠的、基于字节流的传输层通信协议,其完成传输层所指定的功能。传输控制协议(以下简称TCP)层在因特网协议族(Internet Protocal Suite)中处于应用层之下,IP层之上的中间层。


\subsubsection{流媒体}
实时流媒体是随着Internet与多媒体技术的发展而逐渐兴起的一种多媒体网络传输技术,
其典型传输方式为:连续采样的多媒体信息经编码压缩后,
从源端连续、实时地发送到网络,接收端从网络收到部分数据后就可以开始解码播放,
而后续接收的数据则持续不断地存入本地缓存,形成可连续播放的媒体流。其具有连续性、实时性和时序性。

流媒体技术和业务是宽带多媒体技术和业务重的一个重要分支,分为顺序流式传输(Progressive Streaming)和实时流式传输(Real-time Streaming)。顾名思义,顺序流式传输即为按顺序下载,下载文件和观看同时进行,在传输过程重,用户只能观看已下载的部分,想要观看未下载的部分时就会出现卡顿;而实时流式传输则可以在传输期间根据用户的连接速度进行相应的调整,保证媒体信号带宽与网络连接匹配,使用户可以实时观看媒体。

\subsubsection{相关因素的物理关系}
$ \text{1.视频数据量=视频码率} \times\ \text{播放时长} $

$ \text{2.初始缓冲时延=初始缓冲下载数据量} \div\ \text{缓冲时段平均速率--1} $

$ \text{3.初始缓冲下载数据量+播放阶段总时长} \times\ \text{播放阶段平均速率}\geq\ \text{视频数据量--2} $

$ \text{4.不出现卡顿的条件:(视频数据量-初始缓冲下载数据量)} \div\ \text{平均时间} $

$ \text{5. 卡顿时间=(视频数据量-初始缓冲下载数据量)}\div\ \text{播放阶段平均速率-视频播放时间} $
\subsection{模型的建立}
\subsubsection{黑箱模型}
\subsubsection{白箱模型}
\section{模型的求解及结果分析}
\subsection{模型的求解方法}
\subsection{结果分析}

\section{模型的检验}
\section{模型的优缺点及改进方向}
\subsection{模型的优点}
\subsection{模型的缺点}
\subsection{模型的改进方向}
\section{参考文献}
\section{附录}
\end{document}